% -*- coding: utf-8 -*-
\chapter{イントロダクション}
\label{ch:introduction}

朝起きて、ごはんをよそい、味噌汁を作る。
ご飯を食べて、最寄駅まで歩き、職場へ向かう電車に乗る。

ごはんをよそうためには、しゃもじを右手にとり、茶碗を左手に持つ。
炊飯器を空けて、ごはんをかき混ぜる。
かき混ぜたらごはんをしゃもじの上に乗せて、茶碗の上に持っていく。
しゃもじを回転させると、ごはんは茶碗に落ちる。

とても、とても難しいことをやっていると思わないだろうか?
不思議なことに、我々は「ごはんをよそう」と頭にあるだけなのに、そのために必要な行動を列挙し、一つずつ実行していけるのである。

我々が自覚的にはほとんど頭を使わずにこのような計画を立てることが出来るのはなぜだろうか?ごはんをよそうためにお湯を沸かしたり、最寄り駅まで歩いたりする必要はないと分かるのは何故だろうか?
それは我々が無数の選択肢から{\bf 直感} (ヒューリスティック)的に必要そうな行動を絞り込めるからである。

本書で扱う{\bf ヒューリスティック探索}は、先に述べたような直感を駆使し、未来を先読みし、知的に行動を計画する能力をコンピュータに実装しようとする、人工知能研究の一分野である。

%人は様々な問題を探索によって解決している。
%例えば飛行機で成田からロンドンに行く安い/速い方法などを計画するのは探索の一つである。
%一昔前は探索こそが人類の知であるという価値観が広くあり、囲碁、将棋、チェスなどのゲームはそれを競う競技であるとして。
%あるいは囲碁、将棋、チェスなどのゲームも、ある手を選んだ時にどのような局面につながるのかを先読みし、選ぶべき次の一手を探索する。
%このような様々な問題はグラフ探索問題として統合してモデルすることが出来る。
%もちろん、それぞれの問題はそれぞれの特徴があり、それぞれで効率的な解法が異なる。

%\captionlistentry[todo]{Introduction: なんかいい感じの絵}
%{\TODO いい感じの絵}


\section{何故人工知能に探索が必要なのか}
\label{sec:why-search}

% XXX: 「グラフ探索アルゴリズム」が突然出てくる
グラフ探索アルゴリズムは、人工知能分野に限らず情報科学のいろいろな分野で使われる。
本書では、特に人工知能の要素技術としてのグラフ探索アルゴリズムを解説する。

人工知能とは何か、と考えることは本書の主眼ではない。
人工知能の教科書として有名なArtificial Intelligence: Modern Approach \cite{russelln03}では、人工知能研究の目標として以下の4つを掲げている。

\begin{mdframed}[backgroundcolor=gray!10, roundcorner=10pt]
\begin{enumerate}
\item Think Rationally (合理的に考える)
\item Think Humanly (人間的に考える)
\item Act Rationally (合理的に行動する)
\item Act Humanly (人間的に行動する)
\end{enumerate}
\end{mdframed}

グラフ探索アルゴリズムは主にThink Rationallyを実現するための技術である\footnote{探索は4つ全てに強く関係しているが本書は主にThink Rationallyに注視する}。
探索による\define{先読み}{lookahead}{さきよみ}で、最も合理的な手を選ぶことがここでの目的である。

機械学習によるThink Rationallyとの違いは先読みをするという点である。
機械学習は過去の経験を元に合理的な行動を選ぶための技術である。
それに対して、探索では未来の経験を先読みし、合理的な行動を選ぶ。

探索技術の大きな課題・欠点はモデルを必要とする点である。
モデルがないと未来の経験を先読みできない。
例えば将棋で先を読むには、各コマの動き方や、敵の王を詰ますと勝ちであることを知っていなければならない。
加えて、ある局面でどちらがどのくらい有利なのかを評価できないと強いエージェントを作れない。 % エージェントはテクニカルターム
モデルは完璧である必要はないが、ある程度は有用なものでないと先読みがうまく行かなくなる。

%探索は特に学習データを得ることが難しい・不可能なシステムに用いられてきた。
以上のような理由で、探索はモデルが容易に得られる問題において使われてきた。
例えば経路探索問題では地図から距離を推定し、モデルを作ることが出来る。
今後のNASAなどによる宇宙開発でも探索技術が重要であり続けると考えられている。NASAのウェブページを見ると過去と現在の探索技術を用いたプロジェクトがたくさん紹介されている \footnote{\url{https://ti.arc.nasa.gov/tech/asr/planning-and-scheduling/}}。

より賢い行動をするために、探索と機械学習を組み合わせようとする研究もある。
モンテカルロ木探索とディープラーニングを組み合わせてプロ棋士に勝利したAlphaGoなどは、探索と機械学習の組み合わせの強力さを体現している \cite{silver2016mastering}。ここではディープラーニングを使って局面の評価値を学習し、それと探索を組み合わせて評価値が良い局面に繋がる手を選んでいる。
機械学習によってモデルを学習し、それを使って探索をするアプローチもある。先に述べたように探索にはモデルが必要である。例えばAtariでディープラーニングを使って探索のためのモデルを学習する研究がある \cite{oh2015action,silver2016predictron,chiappa2017recurrent}。

このように、探索アルゴリズムは人工知能技術を理解する上で欠かせない分野の一つである。
特に最近大きなブレイクスルーのあった機械学習・深層学習とも強いシナジーを持っているため、これから大きな進展があると期待される分野の一つである。

%Think RationallyとAct Rationallyの間には大きなギャップが存在する。

\begin{comment}
\section{アルゴリズムの概略図}
\label{sec:cheatsheet}

ここでは大まかに問題の特徴に対してどのようなアルゴリズムを最初に試すべきかの概略を示す。
ここで紹介する手法が常に優れているわけではない。が、最初に考慮する価値はあるだろう。
つまり、ここではその問題に当てはまらないとされた手法だからといって有用ではないということはない。
各章を読んでみて、適用可能かを読んでみたい。

\begin{enumerate}
\item 同じ状態に至る経路がたくさんある
	\begin{enumerate}
	\item YES: グラフ探索アルゴリズム
	\item NO: 木探索アルゴリズム (e.g. IDA*)
	\item SOSO: IDA* with transposition table
	\end{enumerate}
	
\item 問題の特徴がある程度分かっておりヒューリスティック関数が作れる
	\begin{enumerate}
	\item YES: ヒューリスティック探索
	\item NO: ブラインド探索
	\end{enumerate}

\item 最適解ではなくある程度良い解なら十分である
	\begin{enumerate}
	\item YES: 局所探索、weighted A*
	\item NO: A*
	\end{enumerate}

\item 最適解と比べてどのくらい良い解かの保証が必要である
	\begin{enumerate}
	\item YES: weighted A*
	\item NO: Greedy best first search
	\end{enumerate}

\item 実行時間が足りない
	\begin{enumerate}
	\item YES: 局所探索、IDA*、weighted A*、並列探索
	\end{enumerate}

\item メモリが足りない
	\begin{enumerate}
	\item YES: 深さ優先探索、IDA*、並列探索、外部メモリ探索、
	\end{enumerate}

\item 状態空間に似たような状態がたくさんある
	\begin{enumerate}
	\item YES: Novelty-based Pruning
	\end{enumerate}

\item 最適解のコストの上界が求められる
	\begin{enumerate}
	\item YES: branch-and-bound
	\end{enumerate}

\item コスト(ノード間距離)は連続値である
	\begin{enumerate}
	\item YES: 二分木によるプライオリティキュー
	\item NO: (離散値) nested bucketによるプライオリティキュー
	\end{enumerate}

\item 分枝数が大きい
	\begin{enumerate}
	\item YES: 遅延重複検出
	\end{enumerate}

\item いろいろな種類の問題を解かなければならない
	\begin{enumerate}
	\item YES: PDDL、自動行動計画問題
	\end{enumerate}

%解を一つではなく全列挙したい
%YES: Symbolic Search?

%あるアクションの後に実行すると有益なアクションが分かっている
%YES: マクロアクション
\end{enumerate}
\end{comment}


\section{本書で扱う内容}
\label{sec:coverage}

% XXX: 状態空間問題って一般的な訳語?
本書で主に扱う問題は\define{状態空間問題}{state-space problem}{じょうたいくうかんもんだい}である。
グラフ探索アルゴリズムは様々な場面で使われるが、この本では特に状態空間問題への応用に注目する。
状態空間問題はゴールに到達するための行動の列、\define{プラン}{plan}{プラン}を発見する問題である。

本書では特に、状態空間問題の中でも
\define{完全情報}{perfect information}{かんぜんじょうほう}かつ\define{決定論的}{deterministic}{けっていろんてき}モデルを取り扱う(\ref{ch:state-space-problem}~\ref{ch:heuristic-search-variants}章
)。

現実世界をそっくりそのままプログラム上で扱うのは困難である。
現実の問題を扱いやすい形式で{\bf モデル化}し、その問題を解くことで現実の問題を解決するのがエンジニアリングである。

世界をどうモデルするべきか判断するのは非常に難しい。
モデルが単純であるほどモデル上の問題は解き易くなるが、単純だが正しいモデルをデザインする・自動生成することは難しい。
その他にも、モデルの理解しやすさや汎用性の観点からモデルの良し悪しは決まる。

完全情報モデルとは、エージェントが世界の状態を全て観察できるモデルである。これは神の目線に立った意思決定のモデルである。
これに対して\define{不完全情報}{partial information}{ふかんぜんじょうほう}モデルでは、エージェントは世界の状態の一部だけを\define{観察}{observation}{かんさつ}することで知ることが出来る。
実世界で動くロボットなどを考えると、不完全情報モデルの方が現実に沿っているが、多くの問題を完全情報モデルで表現できる。

決定論的なモデルではエージェントの行動による世界の状態遷移が一意に(決定論的)に定まる。
一方、非決定論的モデルでは、同じ状態から同じアクションを取ったとしても、世界がどう変化するかは一意には定まらない。
非決定論的モデルにおける探索問題は本書では扱わない。興味があれば強化学習の教科書を読むと良い\cite{sutton1998introduction}。

%モデルを自動生成する方法については\ref{sec:domain-acquisition}章でも簡単に触れるが、本書の主眼ではない。

本書が扱う完全情報決定論的モデルは、上に挙げた中でもシンプルなモデルである。
本書ではこのモデルを対象にしたグラフ探索アルゴリズムを解説する。
%より複雑なモデルに対してはより複雑なアルゴリズムを考える必要がある
%これを不完全情報、非決定論的モデルとすることでより元の問題も表現しやすくなることがあるが、一方でモデルのシンプルさを失うことになる。

% TODO: 関連書籍
\subsection{関連書籍}

ヒューリスティック探索に関連した書籍をいくつか紹介する。
Judea PearlのHeuristic \cite{pearl84}は1984年に出版されたこの分野の古典的名著であり、長く教科書として使われている本である。A*探索の基本的な性質の解析が丁寧に書かれているのでとても読みやすい。また、二人ゲームのための探索に多くの紙面を割いている。
Stefan Edelkamp and Stefan SchrodlのHeuristic Search Theory and Application \cite{edelkamp:2010:hst:1875144}はヒューリスティック探索について辞書的に調べられる本である。2010年出版なのでPearlよりも新しい内容が書かれている。
Stuart Russell and Peter NorvigのArtificial Intelligence \cite{russelln03}は人工知能の定番の教科書である。人工知能に興味がある方はこの本を読むべきである。この本は探索・プランニングだけでなく制約充足問題、機械学習、自然言語処理、画像処理など人工知能のさまざまなテーマを扱っている。
Malik Ghallab, Dana Nau, and Paolo TraversoのAutomated Planning and Acting \cite{ghallab:04}はヒューリスティック探索ではなくプランニングの本である。探索は主にThink Rationallyのための技術だが、探索をロボット制御等に応用するためには考えるだけでなく実際に行動をしなければならない(Act Rationally)。この本では探索をロボットの意思決定に使うための技術的な課題とその解決方法を紹介している。


% TODO: グラフの用語を定義する?

\begin{table}
	\centering
	\begin{tabular}{c | c}
	ユニットコスト状態空間問題 & $P_{u} = (S, A, s_0, T)$ \\
	状態空間問題 & $P = (S, A, s_0, T, w)$ \\
	状態空間グラフ & $G_{u} = (V, E, u_0, T)$ \\
	重み付き状態空間グラフ & $G = (V, E, u_0, T, w)$ \\
	非明示的状態空間グラフ & $G_{i} = (u_0, Goal, Expand, w)$ \\
	状態 & $s, s'$ \\ 
	初期状態 & $s_0$ \\ 
	状態集合 & $S$ \\
	アクション (行動) & $a$ \\
	アクション (行動)集合 & $A$ \\
	ゴール集合 & $T$ \\
	問題グラフ & $G$ \\
	ノード & $u, v, n$ \\
	初期ノード & $u_0$ \\
	ノード集合 & $V$ \\
	エッジ & $e$ \\
	エッジ集合 & $E$ \\
	解 & $\pi$ \\
	コスト関数 & $w$ \\
	実数集合 & $\mathbb{R}$ \\
	ブーリアン集合 & $\mathbb{B}$ \\
	分枝度 & $b$ \\
	深さ & $d$ \\
	最小経路コスト関数 & $g$ \\
	ヒューリスティック関数 & $h$ \\
	プライオリティ関数 & $f$ \\
	最適解のコスト & $c^*$ \\
	オープンリスト & $Open$ \\
	クローズドリスト & $Closed$ \\
	状態変数 & $x$ \\
	命題変数の集合 & $AP$ \\
	適用条件 & pre \\
	追加効果 & add \\
	削除効果 & del \\
%	Expand & Expand \\
%	Goal & Goal \\
%	parent & parent \\
	\end{tabular}
	\caption{表記表}
	\label{notation}
\end{table}
